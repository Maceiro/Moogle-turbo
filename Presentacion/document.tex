
 \documentclass{beamer}
 
 \begin{document}
 	
 	\title{Moogle_Presentation}
 	\author{Luis A. Maceiro}
 	\date{25 de julio, 2023}
 	
 
 \begin{frame}
 	\frametitle{Sistemas de Recuperaci\'on de Informaci\'on }
 	\textbf{Elementos }
 	\begin{itemize}
 		\item Un conjunto de representaciones l\'ogicas de los documentos de la colecci\'on 
 		\item Representaciones l\'ogicas de las necesidades del usuario (consultas)
 		\item Una funci\'on de ranking 
   \end{itemize}
 \end{frame}

\begin{frame}
	\frametitle{Sistema Vectorial}
	 \begin{itemize}
		\item Tanto los documentos de la colecci\'on, como la consulta, se representan mediante vectores del \'algebra lineal n-dimensionales
		\item A cada componente est\'a asociada una palabra del universo, el valor de dicha componente es el peso de la palabra dentro del documento
		\end{itemize}
\end{frame}

\begin{frame}
	\frametitle{Funci\'on de ranking} 
	
	 Sea $t_{i}$ un t\'ermino indexado dentro del documento $d_{j}$ y $w_{ij}$ su peso, entonces dicho documento estar\'a representado por el vector ($w_{1j}, w_{2j},..., w_{nj}$ ) \\
	 
	Sea q la consulta y $w_{iq}$ el peso del t\'ermino $t_{i}$ dentro de la consulta, entonces la consulta puede representarse mediante el vector ($w_{1q}, w_{2q},..., w_{nq}$ )\\
	
	
	 Esta funci\'on est\'a dada por la f\'ormula :\\
	$sim(d_{j}, q)= \frac{ d_{j} \cdot q}{\|d_{j}\| \cdot \|{q}\| } = \frac{ \sum_{i=1}^{n} w_{ij} \cdot w_{iq}}{ \sqrt{\sum_{i=1}^{n} w_{ij}^{2}} \cdot \sqrt{\sum_{i=1}^{n} w_{iq}^{2}}}$ \\
   \end{frame}

  \begin{frame}[fragile] 
	\frametitle{ Clase Document}
	
	\textbf{Variables de instancia}
	\begin{itemize}
		\item string core
		\item string title
		\item \verb|Dictionary<string, int>| tfs             
		\item \verb|Dictionary<string, double>| weigth   
	\end{itemize}



  \textbf{M\'etodos}
  \begin{itemize}
  	\item Constructor para los documentos de la base de datos
  	\item Constructor para la consulta del usuario
  	\item GetIdf
  	\item Similarity
  	\item Sorting
  	\item Tokenizadores
  \end{itemize}

  \end{frame}


  \begin{frame}[fragile]
	\frametitle{Clase Moogle}
	\textbf{M\'etodos}


   \textbf{	GetDocuments}
	\begin{itemize}
		\item Carga los archivos txt desde el directorio en que se encuentran y los transforma en string
		\item Crea instancias de la clase Document con los strings
		\item Llama a GetIdf
	\end{itemize}

    \textbf{Query:}
   \begin{itemize}
   	\item Llama a GetDocuments( preprocesamiento)
   	\item Llama al m\'etodo Sorting de la clase Document (ordena el array de documentos)
   	\item Busca en la query palabras con los operadores \verb|^ y ~|  y modifica los scores seg\'un halla alg\'un operador de cercan\'ia
   	\item La b\'usqueda no arrojar\'a m\'as de 7 resultados. 
   	\item Cada documento que se considere como resultado de la b\'usqueda deber\'a contener a todas las palabras con el operador 
  \end{itemize}

  \end{frame}


  \begin{frame}
   \frametitle{ Operations}
   \textbf{M\'etodos m\'as importantes }
    \begin{itemize}
   	\item Positions
   	\item ChangeScore
   	\item Operations
   	\item Distance
   	\item BestStart
   	\item ObtainValue
   	\item Generadores de palabras a partir de un \'indice determinado siguiendo varios criterios de generac\'ion
   	\end{itemize} 
   \end{frame}
  
   \begin{frame}
   	
   	\textbf{Snippet m\'as \'optimo:} Es aquel que se asemeja m\'as a la consulta /
   	
   	\textbf{ Pasos para cambiar score de los documentos seg\'un operador de cercan\'ia entre dos palabras}
   	\begin{itemize} 
   	\item Se calcula la distancia m\'inima para cada documento particular
   	\item Se halla la mayor de las distancias m\'inimas.
   	\item Se divide la mayor de las distancias m\'inimas entre la distancia m\'inima en el documento actual, se le suma 10, y se aplica el logaritmo en base 10 al resultado.
   	\item Se multiplica el score del documento actual por el valor de variaci\'on obtenido
   	\end{itemize}
   
  \end{frame}


  \begin{frame}
  	\textbf{Clase SuggestionUpdate M\'etodos}
  	\begin{itemize} 
  	\item Suggestion: Recorre todo el vocabulario y compara cada una de las palabras con el string que recibe como par\'ametro utilizando el algoritmo de Levenshtein.
  	\item Levenshtein.
  	\end{itemize}
  
 
   \textbf{  Clase SnippetOperations}
  
  \begin{itemize}
  	\item MakeSnippet
  	\item MakeWords
    \item Generador de palabras
  \end{itemize} 
  
  \end{frame}
	
	
	
 \end{document}